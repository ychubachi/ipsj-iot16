\documentclass[english]{ipsjpapers}
% Set volume, number, etc.
\setcounter{volume}{51}
\setcounter{number}{10}
\setcounter{volpageoffset}{1234}
\received{2010}{7}{17}
\accepted{2010}{9}{17}

% User defined macros
\makeatletter
\let\@ARRAY\@array \def\@array{\def\<{\inhibitglue}\@ARRAY}
\def\<{\begingroup\(\langle\)\it}
\def\>{\/\(\rangle\)\endgroup}
\def\|{\verb|}
\def\cs#1{{\tt\string#1}}
\def\Underline{\setbox0\hbox\bgroup\let\\\endUnderline}
\def\endUnderline{\vphantom{y}\egroup\underline{\box0}\\}
\def\LATEx{\iLATEX{\normalsize\bf A}}
\def\LATex{\iLATEX{\small\bf A}}
\def\LaTeX{\leavevmode\smash{\iLATEX{\sc a}}}
\def\iLATEX#1{L\kern-.36em\raise.3ex\hbox{#1}\kern-.15em
    T\kern-.1667em\lower.7ex\hbox{E}\kern-.125emX}
\def\LATEXe{\ifx\LaTeXe\undefined \LaTeX 2e\else\LaTeXe\fi}
\def\LATExe{\ifx\LaTeXe\undefined \iLATEX\scriptsize 2e\else\LaTeXe\fi}
\def\Quote{\list{}{}\item[]}
\let\endQuote\endlist
\def\LDOTS{$\mathinner{\ldotp\ldotp\ldotp}$}

%\checklines	% Do it to check that baselines are fixed.
\begin{document}%{
% Title
\title[How to Typeset Your Papers in {\protect\LaTeX}]%
	{How to Typeset Your Papers in {\protect\LATEx} (Version 7.2)}
% Definition of Affiliation Labels
\affilabel{KU}{Kyoto University}
\paffilabel{Princeton}{Institute for Advanced Study, Princeton (just joke)}
\affilabel{NTT}{NTT Basic Research Laboratories}
% Author Names
\author{Hiroshi Nakashima\affiref{KU}\affiref{Princeton}\member{8104129}\and
	Yasuki Saito\affiref{NTT}\member{8003547}}

% Contact Address (only for submission, ignored in final version)
\contact{Hiroshi Nakashima\\
	Academic Center for Computing and Media Studies, Kyoto University\\
	Yoshida Honmachi, Sakyo-Ku, Kyoto, 606--8501\\
	phone: (075)753--7457\qquad facsimile: (075)753--7450\\
	email: h.nakashima@media.kyoto-u.ac.jp}

% Absract
\begin{abstract}
This pamphlet is a guide to producing a draft to be submitted to IPSJ Journal
and Transactions and the final camera-ready manuscript of a paper to appear
in the Journal\slash Transactions, using {\LaTeX} and special style
files.  Since the pamphlet itself is produced with the style files, it will
help you to refer its source file which is distributed with the style files.
\end{abstract}

% Output title, etc.
\maketitle

%}{

% Main text starts here.
\section{Introduction}
\footnotetext[1]{The real author is the Editorial Board of the Trans. IPSJ.}

The Information Processing Society of Japan now employs {\LaTeX} to make up
the Journal\slash Transactions for quick and low-cost publishing.  This
means that your {\LaTeX} source file is basically used as the source of the
final printing process.  Therefore, your cooperation is essential for the
publishing of the Journal\slash Transactions inheriting its traditional and
easy-to-read style.

This make-up system, on the other hand, should be also convenient for you,
because it will greatly reduce troubles on proofreading by eliminating
printer's errors inevitable in conventional type-printing systems.
You can easily produce the final version of your paper conforming to the
traditional style using special style files and standard {\LaTeX} commands.
A style file for submission is also available and you can easily switch the
style from submission to final with relatively few changes.
Moreover, the draft
produced by this submission style are much more readable for both you and
referees than those following conventional submission rules.

Although almost everything for final make-up can be done by using standard
{\LaTeX} commands, there are a few additional and essential commands.  Also
there are special rules that are not checked by the style files.  Therefore,
you are requested {\em to read this guide carefully and to follow it
rigidly} in order to make all the people involved in the publishing happy!

%}{

\section{Flow from Submission to Publishing}
\label{sec:Enum}\label{sec:enum}

The process from submission of a paper to publishing the Journal\slash
Transactions featuring it is as follows\footnote[2]{%
%
The following explanation is based on the process of Journal.  Since each
transactions may have its own process, please inquire its detail from each
Editorial Board.}.
%
\begin{Enumerate}%{
\item {\bf Obtaining Style Files}\\
Access the WEB site of IPSJ \|http://www.|\allowbreak\|ipsj.or.jp/| to
download author's kit including style files.  The kit contains the following
files. 
%
\begin{enumerate}%{
\item{\tt ipsjpapers.sty}\mbox{}\\style for final version to make up
\item{\tt ipsjpapers.cls}\mbox{}\\{\LATEXe} style for final version
\item{\tt ipsjdrafts.sty}\mbox{}\\style for drafts to submit
\item{\tt ipsjcommon.sty}\mbox{}\\auxiliary style for final and draft versions
\item{\tt ipsjsort-e.bst}\mbox{}\\Bib{\TeX} style (sorted)%
\footnote[3]{%
The kit also contains \texttt{ipsjsort.bst} and
\texttt{ipsjunsrt.}\allowbreak\texttt{bst} which 
are the counterparts of ``\texttt{-e}'' versions and require a Japanese
\LaTeX{} environment.}
\item{\tt ipsjunsrt-e.bst}\mbox{}\\Bib{\TeX} style (unsorted)%
\footnotemark[3]
\item{\tt esample.tex}\mbox{}\\source of this guide (for final)
\item{\tt desample.tex}\mbox{}\\source of this guide (for draft)
\item{\tt sample.tex}\mbox{}\\Japanese version source of this guide (for final)
\item{\tt dsample.tex}\mbox{}\\Japanese version source of this guide
(for draft)
\item{\tt ebibsample.bib}\mbox{}\\sample of bibliographic data (English)
\item{\tt bibsample.bib}\mbox{}\\sample of bibliographic data (Japanese)
\end{enumerate}%}
%
The kit can be unpacked and read by most of platforms, including UNIX
workstations, Windows (DOS) and Macintosh machines.

\item {\bf Submitting Draft}\\
Prepare the {\LaTeX} source of your draft with the \|draft| option as per
this guide, process it using {\LaTeX} and produce a PDF file.  Since
the style for submission automatically produces an output appropriate for
{\em blind review}, your source file may have commands to give
information which shows who you are and thus cannot appear in the draft.
That is, even if you specify the names and affiliations of authors, their
biographies, and/or acknowledgments to the people and/or organizations
related to you, these items will not appear in your draft version if you use
appropriate commands for them.
Then access the web site
\begin{itemize}\item[]\tt
http://www.ipsj.or.jp/08editt/journal/submit/
\end{itemize}
to register
yourself and to obtain your own URL to submit the PDF file.

\item {\bf Making Final Version}\\
After you receive the notification of acceptance, revise your paper
in accordance with
the comments from referees, and add required omissions from
the draft, such as biography, if any.  The layout of figures and tables
should be fixed.  After that, {\em check your paper again and again to
completely remove description errors}.

\item {\bf Sending Final Version}\\
Send {\em both {\LaTeX} file package and the hard copy} to the IPSJ\@.  The
standard contents of the file package are \|.tex| and \|.bbl|.  If you
include PostScript files and/or special style files, add them into the
package.  Note that {\em you must not split your source into multiple
\|.tex| files}, because it is hard for printers to access multiple files
when they modify your source.  Also carefully make sure that the package
contains all necessary files, especially special style files.

The detail of the file transfer, including its destination and packaging
method, will be instructed to you by the IPSJ secretariat.

\item {\bf Proofreading}\\
The IPSJ may change terms in your paper as per its standard, and the printing
house may modify your source to make it fit the standard printing style.  Even
if they make no changes, the result printed at the printing house may be
different from what you printed because of differences of {\LaTeX}
execution environment.  Therefore, the galley proofs of your paper will be
sent to you to check if those modification and/or differences are
acceptable.  If not, correct errors with red ink.  Note that {\em this
proofreading is not for the correction of your errors} which should have been
corrected before sending the final version.

\item {\bf Typesetting and Publishing}\\
Your paper is typeset, after the correction of the errors you pointed out (if
any), and is published as part of the Journal\slash Transactions.
\end{Enumerate}%}

%}{

\section{{\protect\LATex} Environment}\label{sec:item}

Although a style file, \|ipsjcommon.sty|, has some symbols in Japanese
character set in its last part, you can use the standard (i.e.,
non-Japanese) {\LaTeX} for your English papers because the sytle
autmatically recognizes your environment and lets your {\LaTeX} stop to read
the part it cannot cope with.  One exeception is, however, that you have to use
one of non-Japanese versions of Bib\TeX{} styles \|ipsjsort-e.bst| or
\|ipsjunsrt-e.bst|.

The style files are confirmed to work with the following {\LATEXe}
versions.
%
\begin{itemize}%{
\item[]
{\TeX} 3.141592${}+{}$ {\LaTeXe} 2003/12/01
\end{itemize}%}
%
You may use the styles in either native-mode or {\LaTeX} 2.09 compatible
mode.  Although we expect they will work with older versions, it is strongly
recommended to use the version shown above or later one.

If you still love {\LaTeX} 2.09, do not be afraid to use it because the
sytles are backward compatible.

%}{

\section{How to Use Style Files}
\subsection{General Advice}

The Journal\slash Transactions, as opposed to conference proceedings, have a
traditional and {\em stiff} style.  This makes the style files also {\em
stiff} and strongly restricts the customizability that is one of the useful
features of {\LaTeX}.  For example, you must not change {\em style
parameters}, such as \|\texheight|.  It is not easy to show which
customizations are allowed, but the standard ``Don't tamper with it unless
you are confident'' should work well.

Note that if you do something you should not, {\em you may not have error
messages but simply have ugly results}.

%}{

\subsection{Configuration of Paper}\label{sec:config}

The source file must have the following format.  Underlined parts can be
omitted from draft versions.  Note that a few additional commands shown in
\ref{sec:app-trans} of the Appendix are availabel for a paper included in the
Transactions.
%
\begin{list}{}{\leftmargin.5\leftmargin}\item[]\def\!{\kern-.16667em}\small*\it
\|\documentclass[english]{ipsjpapers}|\footnote{%
Replace it with {\cs\documentstyle} and, if necessary, add auxiliary style
name(s) as its optional argument, when you use {\LATExe} in 2.09-compatible
mode or {\LaTeX} 2.09.}
or\\
\|\documentclass[english,draft]{ipsjpapers}|\rlap{\footnotemark[1]}\\
Specify other option styles if necessary.\\
Specify auxiliary styles by \|\usepackage|.\\
\Underline{\|\setcounter{volume}{|\<volume\>\|}|}\\
\Underline{\|\setcounter{number}{|\<number\>\|}|}\\
\Underline{\|\setcounter{volpageoffset}{|\<first-page\>\|}|}\\
\Underline{\|\received{|\<year\>\|}{|\<month\>\|}{|\<day\>\|}|}\\
\Underline{\|\accepted{|\<year\>\|}{|\<month\>\|}{|\<day\>\|}|}\\
Define your own macros if necessary.\\
\|\begin{document}|\\
\|\title{|\<title\>\|}|\\
\Underline{\|\affilabel{|\<affiliation-label\>\|}{|\<affiliation\>\|}|}\\
\Underline{\mbox{}\qquad\qquad\ldots\ldots\ldots}\\
Declare current affiliation by \|\paffilabel| if necessary.\\
\Underline{\|\author{|\<1st-author\>\|\and|\<2nd-author\>\|\and|\,\LDOTS\|}|}\\
\|\begin{abstract}|\\
\mbox{}\quad\<abstract\>\\
\|\end{abstract}|\\
\|\maketitle|\\
\|\section{|\<heading-of-1st-section\>\|}|\\
\mbox{}\quad $\ldots\ldots\ldots$\\
\mbox{}\quad\<main text\>\\
\mbox{}\quad $\ldots\ldots\ldots$\\
Put acknowledgments here by \|acknowledgment| environment if any.\\
\|\bibliographystyle{ipsjunsrt}| or\\
\|\bibliographystyle{ipsjsort}|\\
\|\bibliography{|\<bib-data-file\>\|}|\\
Put appendices here following \|\appendix| if any.\\
\Underline{\|\begin{biography}|}\\
\Underline{\mbox{}\quad\<biography-of-1st-author\>}\\
\Underline{\mbox\qquad$\ldots\ldots\ldots$}\\
\Underline{\|\end{biography}|}\\
\|\end{document}|
\end{list}

%}{

\subsection{Option Styles}\label{sec:DESC}

The following six standard option styles may be specified as
optional arguments of \|\documentclass| or \|\documentstyle|.
%
\begin{DESCRIPTION}%{
\item[\tt english] for English papers.
\item[\tt landscape] for online publishing\footnote{%
%
This option to typeset in landscape format for online publishing is default.}
%
\item[\tt portrait] for paper publishing.
\item[\tt draft] for draft versions.
\item[\tt technote] for technical notes.
\item[\tt preface] for preface of an issue.
\item[\tt sigrecommended] for a paper recommended by a SIG.
\item[\tt invited] for invited papers.
\end{DESCRIPTION}%}
%
Any (meaningful) combinations of options are acceptable.  The style has
other options to make a non-Journal\slash
Transactions manuscript.  The option \|techrep| is for SIG reports (see
\ref{sec:app-sig} of Appendix) , while
\|private| may be used for your private version (to link it from your own
web page).  With \|private| option, additionally, you may put a copyright
notice to the left top corner of the first page by;
%
\begin{itemize}\item[]%{
\|\copyrightnotice{|\<copyright-notice\>\|}|
\end{itemize}%}
%
as per the IPSJ Copyright Regulation.

If you specify auxiliary style files by \|\usepackage|\footnote{%
Or in the optional argument of \cs{documentstyle} if you use {\LaTeX} 2.09},
%
you must include them into the file package when you send your final version
to IPSJ\@.  However, style files included in {\LATEXe} standard distribusion
(e.g. \|graphicx|) may be omitted.  Note that style files may be
incompatible to the style of the Journal\slash Transactions.

\subsection{Volume, Number, etc.}

If IPSJ notifies you of the volume and number of the issue that your paper is
included in, the first page number of your paper, reception and acceptance
dates, specify them with appropriate commands.  If some (or all) of them are
not notified, you may omit the corresponding commands.  The {\<year\>} should
be a four digit number like 2007, and {\<month\>} should be one or two
digit number like 5 (not May).

%}{

\subsection{Title, Author Names, etc.}\label{sec:Desc}\label{sec:desc}

Describe the title of your paper, author names and affiliations, and
abstract using the commands and environment shown in \ref{sec:config}.  Then
perform \|\maketitle| that automatically puts them at the appropriate position.
In the draft version, the title and abstract are automatically printed onto
separate pages, while author names and affiliations are not printed to make
your paper anonymous.

\begin{Description}
\item[Title]
The title specified by \|\title| is made centered.  Even if the title is
too long to be fit to one line, {\em automatic line break is not performed}.
If your title is long, insert \|\\| into appropriate positions to break
lines.  A multiple line title is first flushed left and then is centered
with respect to the widest line.

The title also appears in the header of odd pages.  If your title is too
long, provide a shortened title for the header to \|\title| as its optional
argument as follows.
%
\begin{quote}
\|\title[|\<for-header\>\|]{|\<title\>\|}|
\end{quote}

\item[Author Name and Affiliation]
Define the affiliation of each author with a label by using \|\affilabel|,
in order from the first author, to have footnotes showing the affiliations with
${\dagger}1$, ${\dagger}2$ and so on.  If two or more authors belong to the
same organization, their affiliation should be declared once.  If an author
moved somewhere after the paper was written and he/she want to show his/her
new affiliation, use \|\paffilabel| to define and to put it with ${\ast}1$,
${\ast}2$, and so on.

The \|\author| argument is a list of author names separated by \|\and|.
Each author name is followed by one or more \|\affiref{|\<label\>\|}| to
attach marks corresponding to labels that have been defined by \|\affilabel|
or \|\paffilabel|.

\item[Abstract]
The abstract of your paper should be given as the contents for the
\|abstract| environment.
\end{Description}

%}{

\subsection{Sectioning}

{\LaTeX} standard commands such as \|\section| and \|\subsection| are
available for sectioning.  The section heading of \|\section| occupies two
lines, while others are put in one line.

For definitions, axioms, theorems, and so on, define and use appropriate
environments with \|\newtheorem|.  Note that the contents of such environments
are not italicized.  If you want have an italicized environment, use
\|\newtheorem*|.

%}{

\subsection{Main Text}\label{sec:desc*}

\begin{description*}
\item[Fixed Baselines]
Each page of the Journal\slash Transactions is formatted with double-column
style.  The printing tradition of double-column requires that a line in the
left column and its neighbor in the right column has the same baseline.  To
meet 
this requirement, the style files carefully control the progression of
baselines when a vertical space is inserted for section titles and so on.
Therefore, {\em you must not use \|\vspace| nor \|\vskip|}.

If you want to check whether baselines progress properly, add the
\|\checklines| command in the preamble to print baselines on which
(ordinary) lines should be located.  This command, however, should be
omitted when you send your source to the IPSJ.

\item[Font Size]
You will see that various size fonts are used in the printed result of your
paper.  Since these fonts are automatically and carefully chosen by the
style files, you are free from headach of selecting proper fonts.  In
fact, it is strongly recommended not to use font-size-changing commands such
as \|\large| and \|\small| in the main text, because they are quite harmful
to the retention of keeping fixed baselines.  If you really want to use
smaller fonts, \|\small| 
or \|\footnotesize|, in order to pack many things in a line, use their {\em
starred} versions, \|\small*| or \|\footnote*|.  They will change the font
size while retaining spaces between baselines the same as \|\normalsize|.
An example of \|\small*| is shown in \ref{sec:config}, and that of
\|\footnotesize*| is in this page.

\item[Overfull and Underfull]
The final result must be free from any overfulls.  It is well known that
almost all overfulls can be avoided by a little effort when describing
sentences.  For example, avoiding long in-text formulas and \|\verb| is
very effective.  However, tricks using \|flushleft| environment, \|\\| or
\|\linebreak| are not recommended, because they cause quite ugly results.

As for underfulls, you will easily get the following warning message
\begin{Quote}\footnotesize*
\|Underfull| \|\hbox| \|(badness 10000)| \|detected|
\end{Quote}
by \|\\| at the end of a paragraph.  This message is also output when you use
\|\\| just before a list-like environment, just before an \|\item|, and at
the end of the environment.  Such underfulls cause ugly empty lines and
flood of warnings that will hide an important error message.
\end{description*}

%}{

\subsection{Formulas}\label{sec:ITEM}

\begin{Itemize}
\item In-text Formulas\\
In-text formulas may be surrounded by any proper math-open\slash close pair,
i.e. \|$| and \|$|, \|\(| and \|\)|, or \|\begin| and \|\end| for \|math|
environment.  Note that tall materials in in-text formulas, such as
\smash{$\frac{a}{b}$} (\|\frac{a}{b}|), are ugly and will disarrange the
baseline progression.

\item Displayed Formulas\\
Displayed formulas {\em must not be surrounded by the pair of \|$$|}.
Instead, use the \|\[| and \|\]| pair or one of the environments
\|displaymath|, 
\|equation| and \|eqnarray|.  These commands\slash environments indent
formulas (not centered) and keep fixed baselines as follows.
\begin{equation}
\Delta_l = \sum_{i=l+1}^L\delta_{pi}.
\end{equation}

\item \|eqnarray| environment\\
For a sequence of two or more related formulas (equations), use the \|eqnarray|
environment to line up them at equal (or unequal) signs, instead of
\|\[|/\|\]| or \|equation| environment.  Note that contents of \|eqnarray|
will not be broken over two pages.  If an \|eqnarray| has many lines and you
want a page break in it, add the option \|[s]| as \|\begin{eqnarray}[s]|.

\item Special Fonts\\
It is strongly recommended to use only standard {\LaTeX} math fonts.
Otherwise, you must report that you are using some special fonts and will be
deeply involved in the dark side of printing process.
\end{Itemize}

%}{

\begin{figure}[t]
\setbox0\vbox{\it
\hbox{\|\begin{figure}[tb]|}
\hbox{\quad \<figure-body\>}
\hbox{\|\caption{|\<caption\>\|}|}
\hbox{\|\label{| $\ldots$ \|}|}
\hbox{\|\end{figure}\|}}
\centerline{\fbox{\box0}}
\caption{Single column figure with caption\\
	explicitly broken by $\backslash\backslash$.}
\label{fig:single}
\end{figure}

\begin{figure}[b]
\begin{minipage}[t]{0.5\columnwidth}
\footnotesize
\setbox0\vbox{
\hbox{\|\begin{minipage}[t]%|}
\hbox{\|  {0.5\columnwidth}|}
\hbox{\|\CaptionType{table}|}
\hbox{\|\caption{| \ldots \|}|}
\hbox{\|\ecaption{| \ldots \|}|}
\hbox{\|\label{| \ldots \|}|}
\hbox{\|\makebox[\textwidth][c]{%|}
\hbox{\|\begin{tabular}[t]{lcr}|}
\hbox{\|\hline\hline|}
\hbox{\|left&center&right\\\hline|}
\hbox{\|L1&C1&R1\\|}
\hbox{\|L2&C2&R2\\\hline|}
\hbox{\|\end{tabular}}|}
\hbox{\|\end{minipage}|}}
\hbox{}
\centerline{\fbox{\box0}}
\caption{Contents of Table \protect\ref{tab:right}.}
\label{fig:left}
\end{minipage}%
\begin{minipage}[t]{0.5\columnwidth}
\CaptionType{table}
\caption{A table built by Fig.\ \protect\ref{fig:left}.}
\label{tab:right}
\makebox[\textwidth][c]{\begin{tabular}[t]{lcr}\hline\hline
left&center&right\\\hline
L1&C1&R1\\
L2&C2&R2\\
\end{tabular}}
\end{minipage}
\end{figure}

\begin{figure*}
\setbox0\vbox{\large
\hbox{\|\begin{figure}*[t]|}
\hbox{\quad\<figure-body\>}
\hbox{\|\caption{|\<caption\>\|}|}
\hbox{\|\label{| $\ldots$ \|}|}
\hbox{\|\end{figure*}|}}
\centerline{\fbox{\hbox to.9\textwidth{\hss\box0\hss}}}
\caption{Double column figure}
\label{fig:double}
\end{figure*}

\subsection{Figures}
A figure fit to one column is specified by the form shown in
\figref{fig:single}.  Note that you must not specify \|h| option.  

The \|\caption| of a figure should be given below of the figure body
together with a \|\label| command.  A long caption will be automatically
broken into two or more lines and centered with respect to the widest line.
You can assist, however, with the line breaking by adding \|\\| to
obtain more beautiful result especially in the case of two-line captions as
shown in \figref{fig:single}.

If you want to rank two or more figures and/or tables in a \|figure| (or
\|table|) environment in order to save space, it is done by enclosing each
figure\slash table and its \|\caption| in a \|minipage| environment as shown
in \figref{fig:left} and \tabref{tab:right}.  Also as exemplified by them
which are in a \|figure| environment, the caption of \tabref{tab:right} is
correctly typeset because the \|minipage| for it has \|\CaptionType{table}|
command to specify the type of caption.  The command of course can be used
with \|figure| argument to give a figure caption.

\Figref{fig:double} shows how to make a double column figure.

You may use any size of fonts as shown in \figref{fig:double}.
Also you may include an encapsulated PostScript file (so called EPS file) as
the body of a figure.  For the inclusion, do;
%
\begin{Quote}
\|\usepackage{graphicx}|
\end{Quote}
%
in the preamble and put \|\includegraphics| command at which you wish to
embed the EPS graphics with its file name (and options if necessary).  If
you use {\LaTeX} 2.09, you have to include \|epsf| in the optional argument
of \|\documentstyle| and use \|\epsfile| for the embedment.  Note that only
the standard fonts shown in Appendix are usable in PostScript files.

You might have noticed that the first reference to \figref{fig:single} is
bold-faced while the second and third are typed in roman fonts.  This font
switching is a rule of the Journal\slash Transactions, and will be
automatically performed if you use \|\figref{|\<label\>\|}| instead of
\|Fig.~\ref{|\<label\>\|}|.  Another rule is that ``Figure'' must be used
instead of ``Fig.''\ if the reference is the first word of a sentence, as
the first reference to \figref{fig:double}.  Unfortunately, this switching
is too hard to do automatically, and you must use \|\Figref{|\<label\>\|}|
in such cases.

%}{

\subsection{Tables}
A table with many rules is not very beautiful.  \Tabref{tab:example} shows
an example of a table with standard style rules.  Note that the uppermost
rule is doubled, and no rules are drawn on the left and right edges.  The
caption should be put above the table.  The default font size in tables is
\|\footnotesize|.  Any reference to a table should be made using
\|\tabref{|\<label\>\|}|\footnote{\cs{\Tabref} is also available but is just
the same as \cs{\tabref}.}.

\begin{table}[b]
\caption{Sections and sub-sections in which list-like environments are used
(example of table)}
\label{tab:example}
\hbox to\hsize{\hfil
\begin{tabular}{l|lll}\hline\hline
&enumerate&itemize&description\\\hline
type-1&	\ref{sec:enum}&	\ref{sec:item}&	\ref{sec:desc}\\
type-2&	---&		\ref{sec:item*}&\ref{sec:desc*}\\
type-3&	\ref{sec:Enum}&	---&		\ref{sec:Desc}\\
type-4&	---&		\ref{sec:ITEM}&	\ref{sec:DESC}\\\hline
\multicolumn{4}{l}{type-1\,: {\tt enumerate}, etc.\quad
	type-2\,: {\tt enumerate*}, etc.}\\
\multicolumn{4}{l}{type-3\,: {\tt Enumerate}, etc.\quad
	type-4\,: {\tt ENUMERATE}, etc.}\\
\end{tabular}\hfil}
\end{table}

%}{

\subsection{Itemizing}\label{sec:item*}

There are four {\em families} of three {\LaTeX} standard itemizing
enviroments, \|enume|{\tt\-}\|rate|, \|itemize| and \|description|, as follows.
%
\begin{itemize*}
\item \|enumerate|, \|itemize|, \|description|\\
Simlar to {\LaTeX}-standard environment except for wider
indentation.  The indentation of \|enumerate| is three times as wide as
\|\parindent|, while those of others are twice.  The \|enumerate| labels
are not {\LaTeX} standard;
%
\begin{quote}
1.\quad (a)\quad i.\quad A.
\end{quote}
%
but have parentheses with small spaces as follows.
%
\begin{quote}
(\,1\,)\quad (\,a\,)\quad (\,i\,)\quad (\,A\,)
\end{quote}

\item \|enumerate*|, \|itemize*|, \|description*|\\
Similar to \|enumerate| etc., but indentation is as wide as
\|\parindent|.

\item \|Enumerate|, \|Itemize|, \|Description|\\
No indentation is performed.

\item \|ENUMERATE|, \|ITEMIZE|, \|DESCRIPTION|\\
Indent only the first line by \|\parindent|.
\end{itemize*}
%
See \tabref{tab:example} to see examples of each environment in this
guide.

%}{

\subsection{Keeping Fixed Baselines}

As described before, every (ordinary) lines in the main text should be
placed on
fixed baselines.  Therefore, if your text has extraordinary tall material
and it shifts other lines from their fixed baselines, enclose the material in
an \|adjustvboxheight| environment.  For example,
%
\begin{adjustvboxheight}
\begin{quote}
\fbox{$\displaystyle\sum_{i=0}^n i$}
\end{quote}
\end{adjustvboxheight}
%
is produced by the following sequence.
%
\begin{Quote}\small*
\|\begin{adjustvboxheight}|\\
\|\begin{quote}|\\
\|\fbox{$\displaystyle\sum_{i=0}^n i$}|\\
\|\end{quote}|\\
\|\end{adjustvboxheight}|
\end{Quote}
%
You will find the line just after the odd thing is on a fixed baseline.

%}{

\subsection{Footnotes}

The command \|\footnote| produces footnotes with marks like \footnote{An
example of footnote.} and \footnotemark, resetting number of
%
% See \footnotetext 74 lines below.
%
footnote marks to one after the page-break.  This automatic adjustment of
footnote marks, however, usually requires {\LaTeX} to be run twice.
(See p.~156 of {\LaTeX}Book\cite{latex}.)

Sometimes, it is preferable to separate a footnote and its mark into
different columns.  You can cope with such a special case using
\|\footnotemark| and \|\footnotetext| commands.

%}{

\subsection{Citations}

There are two styles of citation.  When the citation appears as a word, use
the \|\Cite| command to produce the citation number with normal fonts.
Otherwise, use \|\cite| to have subscripted citations.  For example,
%
\begin{Quote}\tt\raggedright
Goosens explained details of \|\LaTeX|\allowbreak\|\cite{latex}| in 
\|\Cite{companion}|.
\end{Quote}
will produce
\begin{Quote}
Goosens explained details of \LaTeX\cite{latex} in \Cite{companion}.
\end{Quote}
%
as the result.

When three or more texts are cited by \|\Cite| or \|\cite| and their
reference numbers are in series, the first and last numbers are connected by
`--' (en-dash) automatically, as \Cite{book1,book2,booklet1} and
``literatures\cite{latex,inbook1,incollection1,inproceedings1}.''  If
texts cited at once are too numerous to specify them by \|\Cite| or
\|\cite|, use the following {\em multi} versions.
%
\begin{Quote}
\|\multiCite{|\<1st-label\>\|}{|\<last-label\>\|}|\\
\|\multicite{|\<1st-label\>\|}{|\<last-label\>\|}|
\end{Quote}
%
They produce results such as \multiCite{article1}{inproceedings1} and
``literatures\multicite{manual1}{unpublished}.''

%}{

\subsection{References}

References should be arranged in alphabetical or cited order.  It is
strongly recommended to use BiB{\TeX} and style files \|ipsjsort-e.bst|
(alphabetical order) or \|ipsjunsort-e.bst| (cited order) to make references
fit to the traditional style.  You will pick up hints by examining the
sample bibliography file \|ebibsample.|\allowbreak\|bib| and the refereces of
this guide produced by BiB{\TeX} with \|ipsjunsort-e| style.  Please pay
your special attention to \|article| and \|inproceeding|
entries\cite{article3,inproceedings2} with \|doi| fields for papers in
electric journals and digital libraries like;
%
\begin{quote}\tt
doi = "10.2197/ipsjdc.3.14",
\end{quote}
and \|webpage| entries\cite{webpage1,webpage2,webpage3} with \|url| and
\|refdate| (to specify, e.g., Feburary 5th, 2007, being the date on which
you referred to instead of the date on which the page created or modified
most recently) for WEB pages as follows.
%
\begin{quote}\tt
url = "http://search.ieice.org/",\\
refdate = "2007-02-05",
\end{quote}
%
Also remember that you must include \|.bbl| file in the file package,
instead of \|.bib|.

\footnotetext{Another footnote.
This footnote appears right column while its mark is in left
column.  See the source file to know how to do it.}
% See \footnotemark 79 lines above.

If you cannot use Bib{\TeX} and have to make references manually using
\|the|{\tt\-}\|bibliography| environment, observe the references of this guide
carefully and follow its style\footnote{The references of this guide are
produced by {\tt thebiliography} environment to make the source single file,
but the contents are produced by BiB\TeX.}.

%}{

\subsection{Acknowledgments and Appendices}

If you want to acknowledge some people, put your acknowledgments just
before the references and enclose them in the \|acknowledgment| environment.  
Acknowledgments will not be printed in drafts.

Apendices, if any, should be just after the references and \|\appendix|
command.  Sectioning commands produce headings like {\bf A.1}, {\bf A.2} and
so on in apendices.  If you want to make the appendix itself have a title,
give a title to \|\appendix| as its optional argument, like
\|\appendix[|\<title\>\|]|.

%}{

\subsection{Biography}

Biographies of authors must be put just before \|\end{document}| and have
the following format.
%
\begin{Quote}
\|\begin{biography}|\\
\|\author{|\<1st-author's-name\>\|}|\\
\mbox{}\quad\<biography-of-1st-author\>\\
\|\author{|\<2nd-author's-name\>\|}|\\
\mbox{}\quad\<biography-of-2nd-author\>\\
\mbox{}\quad $\ldots\ldots\ldots$ \\
\|\end{biography}|
\end{Quote}
%
The first sentence of each biography must not have subjects and be written
as if its subject is the author's name, e.g. ``was born in 1956.''  The
biographies are not printed in draft versions.

%}{

\subsection{Estimation of Pages}

Roughly speaking, two pages of a draft version are packed into one page of
its final version.  For example, the source of this guide produces a 18-page
draft and 9-page final version, showing the estimation works.

Better estimation, of course, can be obtained by typesetting your draft
using final version style.

%}{

\section{Concluding Remarks}

We don't dream that the style files are perfect, but wish to improve them
with your cooperation and hope you let us know your complainment, comments,
suggessions by e-mail to
%
\begin{Quote}
\|texnicians@ipsj.or.jp|.
\end{Quote}
{\TeX}nical questions also welcome to this address, but other questions on the
Journal\slash Transactions should be received by
\begin{Quote}
\|editt@ipsj.or.jp|.
\end{Quote}

\begin{acknowledgment}
We would like to express our thanks to Sanbi Printing Corp, ULS and Comany,
and all those authors who voluntarily cooperate us in the experimental {\LaTeX}
publishing of the Journal\slash Transactions.
\end{acknowledgment}

%}{

\begin{thebibliography}{10}

\bibitem{companion}
Goossens, M., Mittelbach, F. and Samarin, A.: {\em The LaTeX Companion},
  Addison Wesley, Reading, Massachusetts (1993).

\bibitem{latex}
Lamport, L.: {\em A Document Preparation System {\LaTeX} User's Guide \&
  Reference Manual}, Addison Wesley, Reading, Massachusetts (1986).

\bibitem{article1}
Itoh, S. and Goto, N.: An Adaptive Noiseless Coding for Sources with Big
  Alphabet Size, {\em Trans. IEICE},  Vol.~E74, No.~9, pp.\ 2495--2503 (1991).

\bibitem{article2}
Abrahamson, K., Dadoun, N., Kirkpatrick, D.~G. and Przytycka, T.: A Simple
  Parallel Tree Contraction Algorithm, {\em J. Algorithms},  Vol.~10, No.~2,
  pp.\ 287--302 (1989).

\bibitem{article3}
Yamakami, T.: Exploratory Session Analysis in the Mobile Clickstream, {\em IPSJ
  Digital Courier},  Vol.~3, pp.\ 14--20 (online), \doi{10.2197/ipsjdc.3.14}
  (2007).

\bibitem{book1}
Foley, J.~D. et al.: {\em Computer Graphics --- Principles and Practice},
  System Programming Series, Addison-Wesley, Reading, Massachusetts, 2nd
  edition (1990).

\bibitem{book2}
Chang, C.~L. and Lee, R. C.~T.: {\em Symbolic Logic and Mechanical Theorem
  Proving}, Academic Press, New York (1973).

\bibitem{booklet1}
{Institute for New Generation Computer Technology}: Overview of the Fifth
  Generation Computer Project, distributed in {FGCS'92} (1992).
\newblock (in Japanese).

\bibitem{inbook1}
Knuth, D.~E.: {\em Fundamental Algorithms}, Art of Computer Programming,
  Vol.~1, chapter~2, pp.\ 371--381, Addison-Wesley, 2nd edition (1973).

\bibitem{incollection1}
Schwartz, A.~J.: Subdividing B{\'e}zier Curves and Surfaces, {\em Geometric
  Modeling: Algorithms and New Trends} (Farin, G.~E., ed.), SIAM, Philadelphia,
  pp.\ 55--66 (1987).

\bibitem{inproceedings1}
Baraff, D.: Curved Surfaces and Coherence for Non-penetrating Rigid Body
  Simulation, {\em SIGGRAPH '90 Proceedings} (Beach, R.~J., ed.), Dallas,
  Texas, ACM, Addison-Wesley, pp.\ 19--28 (1990).

\bibitem{inproceedings2}
Nakashima, H. et al.: OhHelp: A Scalable Domain-Decomposing Dynamic Load
  Balancing for Particle-in-Cell Simulations, {\em Proc.\ Intl.\ Conf.
  Supercomputing}, pp.\ 90--99 (online),
  \doi{http://doi.acm.org/10.1145/1542275.1542293} (2009).

\bibitem{manual1}
Adobe Systems Inc.: {\em PostScript Language Reference Manual}, Reading,
  Massachusetts (1985).

\bibitem{mastersthesis1}
Ohno, K.: Efficient Message Communication of Concurrent Logic Programming
  Language KL1 Based on Static Analysis, Master's thesis, Dept. Information
  Science, Kyoto University (1995).

\bibitem{misc1}
Saito, Y. and Nakashima, H.: {{\tt ipsjpapers.sty}} (1995).
\newblock (Style file for Trans. IPSJ distributed to authors.).

\bibitem{phdthesis1}
Weihl, W.: Specification and Implementation of Atomic Data Types, PhD Thesis,
  MIT, Boston (1984).

\bibitem{proceedings1}
Institute for New Generation Computer Technology: {\em Proc. Intl. Conf. on
  Fifth Generation Computer Systems}, Vol.~1 (1992).

\bibitem{WarD:WAM-1}
Warren, D. H.~D.: An Abstract {Prolog} Instruction Set, Technical Report 309,
  Artificial Intelligence Center, SRI International (1983).

\bibitem{unpublished}
{Editorial Board of Trans. IPSJ}: How to Typeset Your Papers in {\LaTeX}
  (Version 1) (1995).
\newblock (distributed to authors).

\bibitem{webpage1}
Kay, A.: Welcome to Squeakland, Squeakland (online),
  \urle{http://www.squeakland.org/community/biography/alanbio.html}
  \refdatee{2007-04-05}.

\bibitem{webpage2}
Nakashima, H.: A {WEB} Page, Kyoto University (online),
  \urle{http://www.para.media.kyoto-u.ac.jp/~nakashima/a.web.page.of.long.url/}
  \refdatee{2010-10-30}.

\bibitem{webpage3}
Nakashima, H.: Another {WEB} Page, Kyoto University (online),
  \urle{http://www.para.media.kyoto-u.ac.jp/~nakashima/a.web.page.of.much.long%
er.url/} \refdatee{2010-10-30}.

\end{thebibliography}

%}{

\appendix
\section{Commands for the Transactions}\label{sec:app-trans}

Each Transactions has its own subtitle, abbreviation code and serial
number.  This information is given by the following command placed before
\|\begin{document}| of the final version source.
%
\begin{itemize}\item[]
\|\transaction{|\<abbrev\>\|}{|\<volume\>\|}{|\<number\>\|}|
\end{itemize}
%
The argument \<abbrev\> must be one of the folloiwngs, while the \<volume\>
and \<number\> of the issue will be notified by the IPSJ or the Editorial
Board of the Transactions.
%
\begin{itemize}%{
\item
\|PRO| (Trans.\ Programming)
\item
\|TOM| (Trans.\ Mathematical Modeling and Its Applications)
\item
\|TOD| (Trans.\ Database)
\item
\|ACS| (Trans.\ Advanced Computing Systems)
\item
\|CVIM| (Trans.\ Computer Vision and Image Media)
\item
\|TBIO| (Trans.\ Bioinformatics)
\item
\|SLDM| (Trans.\ System LSI Design Methodology)
\item
\|CVA| (Trans.\ Computer Vision and Applications)
\end{itemize}%}
%
Note that the \<number\> of the issue does not mean the issue is published
in the \<number\>-th month of a year.  You may be notified about the
\<month\>, to be set to the following \|month| counter, by IPSJ or the
Editorial Board.
%
\begin{itemize}\item[]
\|\setcounter{month}{|\<month\>\|}|
\end{itemize}

Also note that Transactions may have a few local typesetting
convensions shown in the following sections.

%}{

\subsection{Functions for PRO}

Issues of The Transactions on Programming (PRO) not only have regular
papers but also abstracts of the talks given in workshops of SIGPRO.
The file for an abstract consists of the materials from \|\documentclass|
(or \|\documentstyle|) to \|\maketitle| of the format shown in
Section~\ref{sec:config}.  That is, the file does not have a main text.  Note
that the reception and acceptance dates are not necessary but the date of
presentation has to be given by;
%
\begin{itemize}\item[]
\|\presented{|\<year\>\|}{|\<month\>\|}{|\<day\>\|}|
\end{itemize}

%}{

\subsection{Functions for TOM}

Authors of papers included in The Transactions on Mathematical Modeling
and Its Applications (TOM) may be instructed to give the date of reception
of the revised version of the paper.  In this case, the date is given by;
%
\begin{itemize}\item[]
\|\rereceived{|\<year\>\|}{|\<month\>\|}{|\<day\>\|}|
\end{itemize}
%
If the paper has revised twice or more, repeat the command above with each
of the date of revision reception.

%}{

\subsection{Functions for TOD}

The name of the editor in charge for the paper included in The Transactions
on Database (TOD) is given by;
%
\begin{itemize}\item[]
\|\edInCharge{|\<name-of-editor\>\|}|
\end{itemize}

%}{

\subsection{Functions for TBIO}

In order to activate TBIO-specific commands, you should specify \|TBIO|
as an optional argument for \|\documentclass| (or \|documentstyle|).  Since
TBIO accepts English papers only, \|TBIO| option implies \|english| option.
Therefore, you may do;
%
\begin{itemize}\item[]
\|\documentclass[TBIO]{ipsjpaper}|
\end{itemize}
%
without \|english| option.  Then the following commands become available.
%
\begin{itemize}%{
\item
The \<category\> of the paper, \|original|, \|survey| or \|database|, may
be given by;\\
\mbox{}\quad
\|\TBIOpapercategory{|\<category\>\|}|\\
%
in the preamble to put ``{\it Original Paper}'', ``{\it Survey Paper}'' or
``{\it Database\slash Software Paper}'' above the paper title in the first
page. If this command is not given, \|original| is assumed.

\item
The name of the editor in charge for the paper may be given by;\\
\mbox{}\quad
\|\edInCharge{|\<name-of-editor\>\|}|
%
\item
The date of reception of the revised version may be specified by;\\
\mbox{}\quad
\|\rereceived{|\<year\>\|}{|\<month\>\|}{|\<day\>\|}|
%
\end{itemize}
%
Note that the last two commands are optional and thus may be omitted
unless the author is given the information for them and is requested to
specify them.

%}{

\section{How to Make SIG Technical Reports}\label{sec:app-sig}

As SIG technical reports are now published only through IPSJ WEB sites, it
has become the job for each author to produce the PDF manuscript for
publication compliant to the IPSJ standard format.  Typesetting a IPSJ
compliant manuscript, however, is easily done by giving \|techrep|
option to \|\documentclass| command.  One caution for the compliance is
that you have to set the counter \|year| according to the date of the
publication by \|\setcounter| command\footnote{%
%
If the counter is not set in the preamble, you will have a warning message
while the counter will have the value according to the date of the \LaTeX{}
execution.}.
%
Also note that biographies and reception\slash acceptance dates for the
final version of a paper of Journal or Transactions are not printed with
\|techrep| option even if they are specified.

%}{

\begin{biography}
\member{Hiroshi Nakashima}
was born in 1956.  He received his M.E.\ and Ph.D.\ from Kyoto
University in 
1981 and 1991 respectively, and was engaged in research on inference systems
with Mitsubishi Electric Corporation from 1981.
He became an associate professor at Kyoto University in 1992, a professor at
Toyohashi University of Technology in 1997, and a professor at
Kyoto University in 2006.  His
current research interests are the architecture of parallel processing systems
and the implementation of programming languages.  He received the Motooka
award in 1988 and the Sakai award in 1993.  He is a Board Member of IPSJ,
and a member of IEEE-CS, ACM, ALP and TUG.
%
\member{Yasuki Saito}
was born in 1953.  He received his M.S. degree from Univ.\ of Essex, UK in
1978, and M.E. degree from Univ.\ of Tokyo in 1979, respectively.  He has
been working in NTT Corp.\ since 1979 and now is a senior research scientist
of the Basic Research Laboratories of NTT.  Since 1984 until 1985 he had
been a visiting researcher of INRIA, France.  He has been engaging in the
research areas of artificial intelligence (symbol grouping problem),
computer software (Japanese \TeX), cognitive science (learning processes).
He is a member of IPSJ, JSAI, JSSST, JCSS and TUG.
\end{biography}
\end{document}
